\chapter{Conclusões}

Através da construção de um filtro adaptativo para um sistema com três graus de liberdade foi possível avaliar a influência de diversos fatores no projeto do filtro.

Um filtro projetado com um sinal de entrada que possua uma frequência de excitação específica não apresenta boas predições para forçamentos diferentes daquele utilizado no processo adaptativo. Sendo assim, caso se deseje obter melhores predições para um determinado intervalo de frequência, recomenda-se que a frequência da força de entrada seja variada durante o processo de adaptação do filtro.

O ruído branco se mostrou o melhor sinal de entrada para a identificação do sistema em um determinado intervalo de frequência. O filtro construído com esse sinal de entrada e com $ SNR $ alto foi capaz de detectar os três picos referentes às frequências naturais do sistema e também o fenômeno de anti-ressonância. Além disso, o filtro foi capaz de prever com grande exatidão o comportamento do sistema para uma força diferente daquela utilizada no processo de adaptação. Para um $ SNR $ mais baixo o filtro ainda foi capaz de identificar a primeira frequência natural do sistema, mas apresentou resultados ruins para frequências afastadas do primeiro pico e também para a predição a uma força diferente da utilizada no processo de adaptação.

Com relação ao número de amostras utilizados, foi possível perceber que, para o caso de uma excitação em frequência específica, o filtro apresenta uma convergência rápida, sendo necessárias menos de 200 iterações para que o erro se aproximasse de zero. Já para o caso do ruído branco, foi necessária a utilização de um número maior de amostras e também de coeficientes para que a convergência fosse atingida.

Com relação ao fator de convergência, na seção \ref{adapt_F_1} foi possível observar como a sua escolha tem influência no processo de adaptação do filtro. No caso analisado foi possível perceber que um valor baixo diminui a velocidade de adaptação do filtro, mas um valor muito alto pode resultar na não convergência do algoritmo.
